\documentclass[]{article}

% Imported Packages
%------------------------------------------------------------------------------
\usepackage{amssymb}
\usepackage{amstext}
\usepackage{amsthm}
\usepackage{amsmath}
\usepackage{array}
\usepackage{enumerate}
\usepackage{fancyhdr}
\usepackage[margin=1in]{geometry}
\usepackage{graphicx}
\PassOptionsToPackage{hyphens}{url}
\usepackage[hidelinks]{hyperref}
%\usepackage{extarrows}
%\usepackage{setspace}
%\usepackage{xcolor}
\usepackage{color}
%------------------------------------------------------------------------------

% Customize hyperref appearance
%------------------------------------------------------------------------------
\hypersetup{breaklinks=true}
\urlstyle{same}
%------------------------------------------------------------------------------

% Set image path
%------------------------------------------------------------------------------
\graphicspath{ {./images/} }
%------------------------------------------------------------------------------

% Header and Footer
%------------------------------------------------------------------------------
\pagestyle{plain}  
\renewcommand\headrulewidth{0.4pt}                                      
\renewcommand\footrulewidth{0.4pt}                                    
%------------------------------------------------------------------------------

% Title Details
%------------------------------------------------------------------------------
\title{Deliverable \#1 Template : Software Requirement Specification (SRS)}
\author{SE 3A04: Software Design II -- Large System Design}
\date{}
                            
%------------------------------------------------------------------------------

% Document
%------------------------------------------------------------------------------
\begin{document}

\maketitle	
\noindent{\bf Tutorial Number:} T01\\
{\bf Group Number:} G4 \\
{\bf Group Members:} Lukas Buehlmann, David Olejniczak,Vanessa Lai, Saqib Khan, Suzanne Abdullah
% \begin{itemize}
% 	\item Group Member Name (as listed in Avenue)
% 	\item You do not need to use student \#s or macid (keep those private).
% \end{itemize}

\section*{IMPORTANT NOTES}
\begin{itemize}
	\item Be sure to include all sections of the template in your document regardless whether you have something to write for each or not
	\begin{itemize}
		\item If you do not have anything to write in a section, indicate this by the \emph{N/A}, \emph{void}, \emph{none}, etc.
	\end{itemize}
	\item Uniquely number each of your requirements for easy identification and cross-referencing
	\item Highlight terms that are defined in Section~1.3 (\textbf{Definitions, Acronyms, and Abbreviations}) with \textbf{bold}, \emph{italic} or \underline{underline}
	\item For Deliverable 1, please highlight, in some fashion, all (you may have more than one) creative and innovative features. Your creative and innovative features will generally be described in Section~2.2 (\textbf{Product Functions}), but it will depend on the type of creative or innovative features you are including.
\end{itemize}

\newpage
\section{Introduction}
\label{sec:introduction}
This Software Requirements Specification (SRS) provides an overview of the software requirements for the Smart City Environmental Monitoring and Alert System (SCEMAS). SCEMAS is a software application designed to collect, analyze, and visualize environmental data such as air quality, temperature, humidity, and noise levels to support city-level monitoring and decision-making.



\subsection{Purpose}
\label{sub:purpose}
% Begin SubSection
This document outlines the system’s purpose, scope, characteristics of intended users, and the 
functional and non-functional requirements that guide the design and development of the system.
\\\\This document is intended for internal SCEMAS stakeholders, including but not limited to project managers, developers, domain experts, and SCEMAS team members/investors. No prior readings are required.
% End SubSection

\subsection{Scope}
\label{sub:scope}
% Begin SubSection
Smart City Environmental Monitoring and Alert System (SCEMAS) is an environmental monitoring 
software system that allows users to quickly and easily monitor environmental data for metrics 
such as air quality, noise levels, temperature, UV levels, and humidity across geographical 
zones, as well as configure alerts and reminders.\\\\
The software products in the system include a dashboard for real-time data visualization of 
active alerts and geographical maps of sensor locations. Real-time data aggregation will be 
performed within geographical zones to transform data into actionable insights for time-sensitive 
alerts in the system. A read-only REST API for non-sensitive data will be publicly available 
for any user, which uses HTTP methods to ensure safe access and rate limiting to prevent abuse 
and security threats. The system will use MQTT for transmitting all sensor telemetry securely 
through encryption, validating incoming messages for correct format, adherence to the defined 
scheme and validating value ranges. The telemetry stored in a time-series database will not 
store or process PII. Audit log history of all significant events, alert triggers and user 
management will be stored for a minimum of 1 year.\\\\
The system will have role-based access control, which will authorize user-specific access to 
data based on a user’s predefined role: City Operator and System Administrator. The operators 
will primarily use the application to monitor environmental conditions, view raised alerts and 
respond to events quickly using in-app tools. Operators will have write access to change event 
statuses and create alert rules. Administrators will have elevated privileges to approve and 
verify any alert rules being made based on specific thresholds or historical data trends.\\\\
SCEMAS will have a separate public facing dashboard interface that will allow any public user 
to view the visualized non-sensitive data from the REST API.\\\\
Our software will have Service Level Agreements (SLA) to guarantee a dashboard response time 
of 30 seconds upon logging in. A goal of the software is to increase the number of public users 
accessing the REST API to improve data quality and product development.
% End SubSection

\subsection{Definitions, Acronyms, and Abbreviations}
\label{sub:definitions_acronyms_and_abbreviations}
% Begin SubSection
SCEMAS: Smart City Environmental Monitoring and Alert System\\
PIPEDA: Personal Information Protection and Electronic Documents Act. 
RTO: Recovery Time Objective is the maximum amount of time that it should take to restore system operations after a service disruption.\\
PII: Personally Identifiable Information\\
UI: User Interface is the graphical display of the application the users interact with\\
UX: Refers to the user experience using a given user interface\\
MQTT: Message Queuing Telemetry Transport\\
RBAC: Role-Based Access Control\\
IoT: Internet of Things\\
UV: Ultraviolet\\
REST API: Representational State Transfer Application Programming Interface\\
\#-px: Refers to the pixel size of attributes on a user interface, e.g. 14-px sized font\\
% End SubSection

\subsection{References}
\label{sub:references}
% Begin SubSection
\begin{itemize}
	\item[] [1]. K. Erik, “The Responsive Website Font Size Guidelines,” Learn UI Design. Accessed: Feb. 01, 2026. [Online]. Available: \url{https://learnui.design/blog/mobile-desktop-website-font-size-guidelines.html}
	\item[] [2]. Altamira, “Common Screen Sizes for Responsive Web Design.” Accessed: Feb. 01, 2026. [Online]. Available: \url{https://www.altamira.ai/blog/common-screen-sizes-for-responsive-web-design/}
	\item[] [3]. L. Bruton, “User interface guidelines: 10 essential rules to follow - UX Design Institute.” Accessed: Feb. 01, 2026. [Online]. Available: url{https://www.uxdesigninstitute.com/blog/10-user-interface-guidelines/}
	\item[] [4]. “Project Outline for Software Design I” [Online]. Available: \url{https://avenue.cllmcmaster.ca/d2l/le/lessons/727030/topics/5498994 [Accessed: 12-Feb-2026]}
	\item[] [5]. J. Dunn, “Why you need a FAQ page and how to create an effective FAQ template,” KnowledgeBaseBlog ,\url{https://www.knowledgebase.com/blog/why-you-need-a-faq-page-and-how-to-create-an-effective-faq-template/} (accessed Feb. 3, 2026).
	\item[] [6]. M. Hertzem, “Understanding preference: A meta-analysis of user studies”, \url{https://doi.org/10.1016/j.ijhcs.2024.103408} (accessed Feb. 3, 2026).
	\item[] [7]. ILAC, “English proficiency levels: How to determine your level,” \url{https://ilac.com/blog/how-to-determine-english-levels/} (accessed Feb. 3, 2026).
	\item[] [8]. E. Magner, “Digital Signage Font Guide: Chapter 2: Font colors,” RSS, \url{https://www.mvix.com/blog/digital-signage-font-guide-font-colors} (accessed Feb. 3, 2026).
	\item[] [9]. Web.dev (Google), "Why Speed Matters for Your Website and Its Users," 2024. [Online]. Available: \url{https://web.dev/articles/why-speed-matters}. [Accessed: Jan. 31, 2026].
	\item[] [10]. Bodytrak, "The Role of Geolocation Technology in Enhancing Emergency Responses for Worker Safety," May 2024. [Online]. Available: \url{https://bodytrak.co/news/worker-safety-geolocation-technology-emergency-responses/}. [Accessed: Jan. 31, 2026].
	\item[] [11]. UptimeRobot, "What Does 99.9\% Uptime Mean?," June 2024. [Online]. Available: \url{https://uptimerobot.com/blog/what-does-999-uptime-mean/}. [Accessed: Jan. 31, 2026].
	\item[] [12]. IBM, "What is Disaster Recovery?" 2024. [Online]. Available: \url{https://www.ibm.com/topics/disaster-recovery}. [Accessed: Feb. 2, 2026].
	\item[] [13]. Microsoft, "Understand the offline capabilities of IoT Edge devices and gateways," 2024. [Online]. Available: \url{https://learn.microsoft.com/en-us/azure/iot-edge/offline-capabilities}. [Accessed: Feb. 2, 2026].
	\item[] [14]. GeeksforGeeks, “Graceful Degradation in Distributed Systems,” 2025. [Online]. Available: \url{https://www.geeksforgeeks.org/system-design/graceful-degradation-in-distributed-systems/}. [Accessed: Feb. 3, 2026].
	\item[] [15]. Netdata, "What is Database Concurrency?" 2024. [Online]. Available: \url{https://www.netdata.cloud/academy/what-is-database-concurrency/}. [Accessed: Feb. 3, 2026].
	\item[] [16]. StatCounter Global Stats, “Desktop Operating System Market Share Worldwide.” Accessed: Feb. 01, 2026. [Online]. Available: \url{https://gs.statcounter.com/os-market-share/desktop/worldwide/}
	\item[] [17]. P. Solutions, “How to minimize downtime during your control system upgrade,” Process Solutions, Inc., \url{https://processsolutions.com/how-to-minimize-downtime-during-your-control-system-upgrade/} (accessed Feb. 3, 2026).
	\item[] [18]. K. O’Brien, “What is application monitoring?,” IBM, \url{https://www.ibm.com/think/topics/application-monitoring} (accessed Feb. 3, 2026).
	\item[] [19]. R. Davydov, “How to build scalable, future-proof custom software,” Built In, \url{https://builtin.com/articles/build-future-proof-software} (accessed Feb. 3, 2026).
	\item[] [20]. R. S. Sandhu, E. J. Coyne, H. L. Feinstein and C. E. Youman, "Role-based access control models," in Computer, vol. 29, no. 2, pp. 38-47, Feb. 1996, doi: 10.1109/2.485845.
	\item[] [21]. Canadian Centre for Cyber Security, “Annex 3A: Security control catalogue,” Guidance on the Management of IT Security Risks (ITSG-33). [Online]. Available: \url{https://www.cyber.gc.ca/en/guidance/annex-3a-security-control-catalogue-itsg-33}. [Accessed: Feb. 10, 2026].
	\item[] [22]. Government of Canada, "Official Languages Act (R.S.C., 1985, c. 31 (4th Supp.))," Justice Laws Website. [Online]. Available: \url{https://laws-lois.justice.gc.ca/eng/acts/o-3.01/}. [Accessed: Jan. 31, 2026].
	\item[] [23]. Government of Canada, "Criminal Code (R.S.C., 1985, c. C-46), s. 319," Justice Laws Website. [Online]. Available: \url{https://laws-lois.justice.gc.ca/eng/acts/c-46/section-319.html}. [Accessed: Jan. 31, 2026].
	\item[] [24]. Criminal Code, R.S.C. 1985, c. C-46, s. 372. [Online]. Available: \url{https://laws-lois.justice.gc.ca/eng/acts/c-46/section-372.html}
	\item[] [25]. Government of Canada, "Criminal Code (R.S.C., 1985, c. C-46), s. 184," Justice Laws Website. [Online]. Available: \url{https://laws-lois.justice.gc.ca/eng/acts/c-46/section-184.html}. [Accessed: Jan. 31, 2026].
	\item[] [26]. Government of Canada, "Personal Information Protection and Electronic Documents Act (S.C. 2000, c. 5)," Justice Laws Website. [Online]. Available: \url{https://laws-lois.justice.gc.ca/eng/acts/p-8.6/}. [Accessed: Jan. 31, 2026].
	\item[] [27]. Government of Ontario, "Integrated Accessibility Standards, O. Reg. 191/11," ServiceOntario e-Laws. [Online]. Available: \url{https://www.ontario.ca/laws/regulation/110191}. [Accessed: Jan. 31, 2026]
	\item[] [28]. Government of Ontario, "Municipal Freedom of Information and Protection of Privacy Act, R.S.O. 1990, c. M.56," ServiceOntario e-Laws. [Online]. Available: \url{https://www.ontario.ca/laws/statute/90m56}. [Accessed: Jan. 31, 2026].
	\item[] [29]. Government of Canada, "Canadian Environmental Protection Act, 1999 (S.C. 1999, c. 33)," Justice Laws Website. [Online]. Available: \url{https://laws-lois.justice.gc.ca/eng/acts/c-15.31/}. [Accessed: Jan. 31, 2026].
	\item[] [30]. World Wide Web Consortium, "Web Content Accessibility Guidelines (WCAG) 2.1," W3C Recommendation, June 5, 2018. [Online]. Available: \url{https://www.w3.org/TR/WCAG21/}. [Accessed: Jan. 31, 2026].
\end{itemize}
% End SubSection

\subsection{Overview}
\label{sub:overview}
% Begin SubSection
Section 2 discusses the overall product description including the product functions, user 
characteristics, constraints, assumptions, and the perspective of the product relative to 
existing projects. Section 3 contains a use case diagram for the system. Section 4 highlights 
the functional requirements including the use cases organized by business event, different 
viewpoints for each event and associated scenarios, and a global scenario for each business 
event. Section 5 discusses the non-functional requirements including the look and feel 
requirements, usability and human requirements, performance requirements, operational 
requirements, maintainability, security, and political requirements.
% End SubSection

% End Section

\newpage
\section{Overall Product Description}
\label{sec:overall_description}
% Begin Section

\subsection{Product Perspective}
\label{sub:product_perspective}
% Begin SubSection
SCEMAS is a smart city monitoring dashboard developed for environmental monitoring of the 
average smart city built for a web application. Related products include Siemens MindSphere, 
AxxonSoft (Safe City) and Cisco Kinetic for Cities, where city-wide data is monitored from 
various IoT devices based on geographic zones. However SCEMAS differentiates itself by 
focusing on unifying all environmental data into one centralized dashboard and database 
with the ability to respond to events via built-in tools rather than just observing them. 
The product will facilitate various roles of access, allowing users to be created, edited, 
and removed, while keeping extensive historical logs of all user actions, alerts, and 
collected data for auditing similar to the compliance standards found in the related 
products. Various alerts and reminders can be created based on geographical zones and 
real-time aggregated data like in AxxonSoft.\\\\
The product will interact with the larger system of distributed IoT sensors utilizing the 
MQTT protocol to protect the data and ensure system privacy and security. SCEMAS will act 
as a central ingestion hub, receiving high-volume telemetry streams from sensors to detect 
threshold violations in real-time. The product will also utilize real-world mapping 
software such as OpenStreetMap or GoogleMaps to visualize these telemetry streams. Similar 
to AxxonSoft map view, SCEMAS will overlay live heatmaps and alert markers on the geography, 
allowing City Officials to pinpoint the exact location of environmental hazards and acknowledge 
them directly from the map interface.\\\\
The product will have a sub-system to export data for public use, using a secure getaway 
allowing the city to share non-sensitive environmental information to an external public 
facing website and a read-only API. This website will be open to the public and will have 
a heatmap on limited metrics. This ensures City Officials have full command-and-control 
capabilities of public facing data. 

\begin{figure}[h!]
    \centering
    \includegraphics[width=0.75\linewidth]{A04_System.png}
    \caption{SCEMAS System Diagram}
    \label{fig:system_diagram}
\end{figure}
% End SubSection

\subsection{Product Functions}
\label{sub:product_functions}
% Begin SubSection
The system functionality will be focused on both data collection and processing done 
automatically as well as interactions between users and the system. At a high level, the 
system is responsible for collecting environmental data, storing this data, calculating 
aggregations of data series by city zone, and comparing stored alert configurations to 
present data. The city operators and administrators can interact with this system through 
a dashboard interface while public users can use a public facing API that provides different 
functionality from the city dashboard. An important property of the system is that it has 
functionality to authenticate users and limit the access of different users based on their role. 

\begin{center}
\begin{tabular}{ |m{2in}|m{4.5in}| } 
 \hline
 \bf{Function} & \bf{Description} \\ 
 \hline
 Environmental Data Collection & The system collects data from many sensors throughout a city. The system ensures the incoming data is in the correct format, within bounds, and valid before accepting it. \\ 
 \hline
 Data Storage and Management & Sensor data is stored for real-time and historical use, organized by time and stored by city zone. \\ 
 \hline
 Data Processing and Aggregation & Data is aggregated over time and by city zone for both private use and public access depending on the sensitivity of the data.  \\ 
 \hline
 Alert Detection and Management & Administrators can manage and add new alert rules. Alerts occur when real time data triggers the rule of an alert. Alerts are logged and notifications are sent to the dashboard \\ 
 \hline
 Dashboard for City Operators & The dashboard is a secure interface for city operators to view real time data, current alerts, and view trends.  \\ 
 \hline
 Public Data Access Interface & A read-only public interface is exposed which only allows access to aggregated, non-sensitive environmental data. \\ 
 \hline
 RBAC Management & The system supports Role-Based Access Control to differentiate administrators from city operators.  \\ 
 \hline
 Authentication & All user sign-ins are authenticated and all data in transit is encrypted. All significant events, such as alert triggers and administrative changes, are logged by the system.  \\ 
 \hline
\end{tabular}
\end{center}
% End SubSection

\begin{figure}[h!]
    \centering
    \includegraphics[width=0.75\linewidth]{A04_State.png}
    \caption{SCEMAS State Diagram}
    \label{fig:state_diagram}
\end{figure}

\subsection{User Characteristics}
\label{sub:user_characteristics}
% Begin SubSection
As the app will have different interfaces for different types of users, it is important for us 
to define the distinct needs and abilities of each category of user.\\\\
\textbf{City Operators}\\
City operators will be responsible for monitoring the application's dashboard to put out alerts for events.

\begin{enumerate}
	\item \textbf{Educational level:} Secondary school education
		\begin{itemize}
			\item City operators are expected to have at least a high-school level of education to properly interpret the data on the dashboard.
		\end{itemize}
	\item \textbf{Experience:} Prior training in using the system
		\begin{itemize}
			\item City operators are expected to be trained in how to use the dashboard prior to starting their role. Prior experience in using a similar dashboard would also be acceptable experience.
		\end{itemize}
	\item \textbf{Technical experience:} Average computer knowledge
		\begin{itemize}
			\item City operators are expected to have a medium level of expertise in using computers, including an understanding of file systems and more complex graphical user interfaces.
		\end{itemize}
\end{enumerate}
\textbf{City Residents}\\
City residents will view city-wide alerts notifications from a different interface than city operators.
\begin{enumerate}
	\item \textbf{Educational level:} Basic reading and comprehension ability
		\begin{itemize}
			\item City residents fall into a wide range of age and education levels, so in order to best accommodate the target audience, the application should be accessible to those with only a primary school education level. 
		\end{itemize}
	\item \textbf{Experience:} None
		\begin{itemize}
			\item City residents should be able to learn to use the app without having any prior experience with it, or similar products. 
		\end{itemize}
	\item \textbf{Technical experience:} Minimal understanding of computers
		\begin{itemize}
			\item City residents are expected to have a minimal understanding of how computers work in order to launch the application, as well as some understanding of common app navigation methods (e.g. basic menus, buttons)
		\end{itemize}
\end{enumerate}
\textbf{Application Administrators}\\
Administrators will be able to define alert rules for the city operator dashboard.
\begin{enumerate}
	\item \textbf{Educational level:} Post-Secondary education
		\begin{itemize}
			\item The administrators of the system are expected to have completed a four year program at a university or college relating to city management or administrative systems. 
		\end{itemize}
	\item \textbf{Experience:} Prior training in using the system
		\begin{itemize}
			\item The application administrators of the system are expected to have prior training in using the system, as well as prior experience in using similar software.
		\end{itemize}
	\item \textbf{Technical experience:} Mid-High level of computer knowledge
		\begin{itemize}
			\item System admins are expected to have a mid to high understanding of computers, and should have a good understanding of file systems, complex system interfaces, trouble-shooting, and potentially command-line interfacing.
		\end{itemize}
\end{enumerate}
The app should also be accessible to individuals with varying levels of visual, auditory and mobility impairments on all user-facing-interfaces to ensure it is usable for all members of the general population. 
% End SubSection

\subsection{Constraints}
\label{sub:constraints}
% Begin SubSection
The following are constraints which may limit the developer's options and approach:
\begin{itemize}
	\item[] \textbf{Communication Protocol:} The system must accept and send all environmental telemetry data using the MQTT protocol. This is mandated by the existing IoT infrastructure already deployed in smart city environments. The system cannot use alternative protocols such as HTTP.
	\item[] \textbf{Regulation Compliance:} The system must comply with all data protection and privacy regulations, such as the PIPEDA in Canada. 
	\item[] \textbf{Public Data Access:} Only non-sensitive environmental data should be exposed through the public interface. Any sensitive data must be restricted for authorized users. This constraint stems from the privacy-by-design requirement specified in the system design.
	\item[] \textbf{Budget Constraint:} The system must be developed and deployed with zero monetary budget. All software tools, cloud services, databases, and third-party APIs must utilize free-tier offerings.
	\item[] \textbf{Fixed Timeline:} The system must be developed within the time allocated. This fixed deadline constrains the scope of features that can be implemented and requires prioritization of core functionality over advanced features.
\end{itemize}
% End SubSection

\subsection{Assumptions and Dependencies}
\label{sub:assumptions_and_dependencies}
% Begin SubSection
\begin{enumerate}
	\item We are a private company being contracted by the city government to create this platform. 
	\item The project is assumed to be funded by the Government of Canada.
	\item The system is assumed to be limited to operation within one city.
	\item It is assumed that there is valid data flowing into the system from various IoT devices for Business Events to be successful (as stated in the Precondition for BE1).
	\item It is assumed that accompanying training modules will be provided for City Operators, and they can read the system documentation.
\end{enumerate}
% End SubSection

\subsection{Apportioning of Requirements}
\label{sub:apportioning_of_requirements}
% Begin SubSection
N/A
% End SubSection

% End Section

\newpage
\section{Use Case Diagram}
\label{sec:use_case_diagram}
% Begin Section
\begin{figure}[h!]
    \centering
    \includegraphics[width=0.75\linewidth]{A04_UseCase.png}
    \caption{SCEMAS Use-Case Diagram}
    \label{fig:use_case_diagram}
\end{figure}
%In this section, select the most important Business Event that your system responds to and give its use case diagram.  Only one use case diagram is needed.  Give a brief textual description of the use case without repeating what is in the scenarios of the corresponding Business Event.

%
%
%
%This section should provide a use case diagram for your application. 
%\begin{enumerate}[a)]
%	\item Each use case appearing in the diagram should be accompanied by a text description. 
%\end{enumerate}
%% End Section

\newpage
\section{Highlights of Functional Requirements}
\label{sec:functional_requirements}
% Begin Section
\begin{itemize}
	\item Specify all use cases (or other scenarios triggered by other events), organized by Business Event. 
	\item For each Business Event, show the scenario from every Viewpoint. You should have the same set of Viewpoints across all Business Events. If a Viewpoint doesn't participate, write N/A so we know you considered it still. You can choose how to present this - keep in mind it should be easy to follow. 
	\item At the end, combine them all into a Global Scenario.
	%\item Specify the "use cases" (or other triggering events) organized by Business Event. (The Global Scenario is what you might think of as a use case). Be sure to consider Business Events that aren't just triggered by users with goals (e.g. something happens in the environment that your system needs to respond to)
	\item Your focus should be on what the system needs to do, not how to do it. Specify it in enough detail that it clearly specifies what needs to be accomplished, but not so detailed that you start programming or making design decisions.
	\item Keep the length of each use case (Global Scenario) manageable. If it's getting too long, split into sub-cases.
	\item You are \emph{not} specifying a complete and consistent set of functional requirements here. (i.e. you are providing them in the form of use cases/global scenarios, not a refined list). For the purpose of this project, you do not need to reduce them to a list; the global scenarios format is all you need.
	\item Red text below is just to highlight where you need to insert a scenario - don't actually write it all in red.
\end{itemize}

\noindent {\bf Main Business Events:} List out all the main business events you are presenting. If you sub-divided into smaller ones, you don't need to include the smaller ones in this list.\\

\noindent {\bf Viewpoints:} List out all the viewpoints you will be considering.\\

\noindent {\bf Interpretation:} Specify any liberties you took in interpreting business events, if necessary.\\

\begin{enumerate}[{\bf BE1.}]
	\item Business Event Name \#1
		\begin{enumerate}[{\bf VP1.}]
			\item Viewpoint Name \#1 \\
				\textcolor{red}{Insert Scenario Here}
			\item Viewpoint Name \#2 \\
				\textcolor{red}{Insert Scenario Here}
		\end{enumerate}
		{\bf Global Scenario:}\\
		\textcolor{red}{Insert Scenario Here}
	\item Business Event Name \#2
	\begin{enumerate}[{\bf VP1.}]
		\item Viewpoint Name \#1 \\
		\textcolor{red}{Insert Scenario Here}
		\item Viewpoint Name \#2 \\
		\textcolor{red}{Insert Scenario Here}
	\end{enumerate}
	{\bf Global Scenario:}\\
	\textcolor{red}{Insert Scenario Here}
\end{enumerate}

%	Below, we organize by Business Event.
%	\begin{enumerate}[{BE}1.]
%		\item Business Event name
%		\begin{enumerate}[{VP1}.1]
%			\item Viewpoint name \newline
%			\noindent\fbox{%
%				\parbox{0.5\textwidth}{%
%					\begin{itemize}
%						\item {\bf $S_{1}$:} Initial response of the system to the Business Event
%						\item {\bf $E_{1}$:}  Reaction of the environment to $S_{1}$
%						\item {\bf $S_{2}$:}  Response of the system to $E_{1}$
%						\item {\bf $E_{2}$:}  Reaction of the environment to $S_{2}$
%						\item[] $\cdots$
%						\item {\bf $S_{n}$:}  Response of the system to $E_{(n-1)}$
%						\item {\bf $E_{n}$:}  Reaction of the environment to $E_{(n-1)}$
%						\item {\bf $S_{(n+1)}$:} Final response of the system concluding its function regarding the Business Event
%					\end{itemize}
%				}%
%			}
%			\item Viewpoint name\newline
%			\noindent\fbox{%
%				\parbox{0.5\textwidth}{%
%					\begin{itemize}
%						\item {\bf $S_{1}$:} Initial response of the system to the Business Event
%						\item {\bf $E_{1}$:}  Reaction of the environment to $S_{1}$
%						\item {\bf $S_{2}$:}  Response of the system to $E_{1}$
%						\item {\bf $E_{2}$:}  Reaction of the environment to $S_{2}$
%						\item[] $\cdots$
%						\item {\bf $S_{k}$:}  Response of the system to $E_{(k-1)}$
%						\item {\bf $E_{k}$:}  Reaction of the environment to $E_{(k-1)}$
%						\item {\bf $S_{(k+1)}$:} Final response of the system concluding its function regarding the Business Event
%					\end{itemize}
%				}%
%			}
%			\item \dots
%			\item \dots
%			\item \dots
%			\item[\dots]
%		\end{enumerate}	
%		\item[] {\bf Global Scenario of {\it Business Event Name}:} It is the scenario corresponding to the integration of all the above scenarios from the different Viewpoints of the Business Event BE1.\newline
%		\noindent\fbox{%
%			\parbox{0.5\textwidth}{%
%				\begin{itemize}
%					\item {\bf $S_{1}$:} Initial response of the system to the Business Event
%					\item {\bf $E_{1}$:}  Reaction of the environment to $S_{1}$
%					\item {\bf $S_{2}$:}  Response of the system to $E_{1}$
%					\item {\bf $E_{2}$:}  Reaction of the environment to $S_{2}$
%					\item[] $\cdots$
%					\item {\bf $S_{m}$:}  Response of the system to $E_{(m-1)}$
%					\item {\bf $E_{m}$:}  Reaction of the environment to $E_{(m-1)}$
%					\item {\bf $S_{(m+1)}$:} Final response of the system concluding its function regarding the Business Event
%				\end{itemize}
%			}%
%		}	
%		%\end{enumerate}
%		\item Business Event name
%		\begin{enumerate}[{VP1}.1]
%			\item Viewpoint name \newline
%			\noindent\fbox{%
%				\parbox{0.5\textwidth}{%
%					\begin{itemize}
%						\item {\bf $S_{1}$:} Initial response of the system to the Business Event
%						\item {\bf $E_{1}$:}  Reaction of the environment to $S_{1}$
%						\item {\bf $S_{2}$:}  Response of the system to $E_{1}$
%						\item {\bf $E_{2}$:}  Reaction of the environment to $S_{2}$
%						\item[] $\cdots$
%						\item {\bf $S_{n'}$:}  Response of the system to $E_{(n'-1)}$
%						\item {\bf $E_{n'}$:}  Reaction of the environment to $E_{(n'-1)}$
%						\item {\bf $S_{(n'+1)}$:} Final response of the system concluding its function regarding the Business Event
%					\end{itemize}
%				}%
%			}
%			\item Viewpoint name\newline
%			\noindent\fbox{%
%				\parbox{0.5\textwidth}{%
%					\begin{itemize}
%						\item {\bf $S_{1}$:} Initial response of the system to the Business Event
%						\item {\bf $E_{1}$:}  Reaction of the environment to $S_{1}$
%						\item {\bf $S_{2}$:}  Response of the system to $E_{1}$
%						\item {\bf $E_{2}$:}  Reaction of the environment to $S_{2}$
%						\item[] $\cdots$
%						\item {\bf $S_{k'}$:}  Response of the system to $E_{(k'-1)}$
%						\item {\bf $E_{k'}$:}  Reaction of the environment to $E_{(k'-1)}$
%						\item {\bf $S_{(k'+1)}$:} Final response of the system concluding its function regarding the Business Event
%					\end{itemize}
%				}%
%			}
%			\item \dots
%			\item \dots
%			\item \dots
%			\item[\dots]
%		\end{enumerate}	
%		\item[] {\bf Global Scenario of {\it Business Event Name}:} It is the scenario corresponding to the integration of all the above scenarios from the different Viewpoints of the Business Event BE2.\newline
%		\noindent\fbox{%
%			\parbox{0.5\textwidth}{%
%				\begin{itemize}
%					\item {\bf $S_{1}$:} Initial response of the system to the Business Event
%					\item {\bf $E_{1}$:}  Reaction of the environment to $S_{1}$
%					\item {\bf $S_{2}$:}  Response of the system to $E_{1}$
%					\item {\bf $E_{2}$:}  Reaction of the environment to $S_{2}$
%					\item[] $\cdots$
%					\item {\bf $S_{m'}$:}  Response of the system to $E_{(m'-1)}$
%					\item {\bf $E_{m'}$:}  Reaction of the environment to $E_{(m'-1)}$
%					\item {\bf $S_{(m'+1)}$:} Final response of the system concluding its function regarding the Business Event
%				\end{itemize}
%			}%
%		}		
%	\end{enumerate}

%End Section

\section{Non-Functional Requirements}
\label{sec:non-functional_requirements}

% Begin Section
\subsection{Look and Feel Requirements}
\label{sub:look_and_feel_requirements}
% Begin SubSection

\subsubsection{Appearance Requirements}
\label{ssub:appearance_requirements}
% Begin SubSubSection
\begin{enumerate}[{LF-A}1. ]
	\item The application should use a distinct colour scheme that is in adherence with the companies brand colours.\newline
	{\bf Rationale:} The application's UI should be consistent and professional to establish a sense of trust in the product's users [1].
	\item All text in the application should be at least 14-px in size.\newline
	{\bf Rationale:} The text in the application should follow font-size conventions to be large enough so it is easy to read for a wide range of individuals with varying vision levels [1].
	\item The background colours of pages should contrast with the interactive elements (alert notifications, buttons, icons) directly on top of them.\newline
	{\bf Rationale:} The interactive elements on the page should stand out from the background elements in order to indicate what can be interacted with on the page [1].

\end{enumerate}
% End SubSubSection

\subsubsection{Style Requirements}
\label{ssub:style_requirements}
% Begin SubSubSection
\begin{enumerate}[{LF-S}1. ]
	\item The application's interface should be built for a resolution of 1920x1080 and be compatible with various screen sizes across desktop devices.\newline
	{\bf Rationale:} As the interface will be used across many different desktop devices with varying screen sizes, the interface's size must be adaptable so all users can view the application as intended. 1920x1080 has been selected as the default size as it is the most common resolution on modern desktop devices [2].
	\item The application's interface must utilize modern conventions of professional software UI/UX to achieve a professional and minimalist design.\newline
	{\bf Rationale:} Adhering to commonly used UI/UX design principles will make the application look polished and professional, instilling trust in the user. Using well-known conventions of web design will also help the application to be more user-friendly [3].
	\item All main pages of the application interface should be accessible from one menu (e.g. a navigation bar).\newline
	{\bf Rationale:} The application should be simple to navigate, and users should not have to search through many menus to find crucial features [4].
\end{enumerate}
% End SubSubSection

% End SubSection

\subsection{Usability and Humanity Requirements}
\label{sub:usability_and_humanity_requirements}
% Begin SubSection

\subsubsection{Ease of Use Requirements}
\label{ssub:ease_of_use_requirements}
% Begin SubSubSection
\begin{enumerate}[{UH-EOU}1. ]
	\item The system must include a help menu for users, including a frequently asked questions page, and tutorial pages for how to navigate.\newline
	{\bf Rationale:} A frequently asked questions page will be a helpful resource for users by providing answers to repetitive inquiries [5].
\end{enumerate}
% End SubSubSection

\subsubsection{Personalization and Internationalization Requirements}
\label{ssub:personalization_and_internationalization_requirements}
% Begin SubSubSection
\begin{enumerate}[{UH-PI}1. ]
	\item The system must allow users to configure personal preferences in the account settings. Features that must be selectable are light or dark mode and accessibility routes highlighted on the maps for users with disabilities.\newline
	{\bf Rationale:} Enabling user-defined preferences improves the overall user experience and decreases task time [6]. The software will provide equitable utility for different users, making the application more inclusive.
\end{enumerate}
% End SubSubSection

\subsubsection{Learning Requirements}
\label{ssub:learning_requirements}
% Begin SubSubSection
\begin{enumerate}[{UH-L}1. ]
	\item The system must allow any third-party developer to successfully request and interpret environmental data within a two-hour time frame.\newline
	{\bf Rationale:} A short learning curve ensures the API is intuitive and well-documented for the public. A short onboarding period encourages adoption of the REST API and improves developer productivity [4].
\end{enumerate}
% End SubSubSection

\subsubsection{Understandability and Politeness Requirements}
\label{ssub:understandability_and_politeness_requirements}
% Begin SubSubSection
\begin{enumerate}[{UH-UP}1. ]
	\item The proficiency requirements for understanding any language being used in the system shall not exceed the B1 language proficiency level.\newline
	{\bf Rationale:} The system application should be understandable to a wide range of people and a B1 proficiency level indicates the user can handle everyday situations and express opinions [7].
\end{enumerate}
% End SubSubSection

\subsubsection{Accessibility Requirements}
\label{ssub:accessibility_requirements}
% Begin SubSubSection
\begin{enumerate}[{UH-A}1. ]
	\item The system must have a setting to turn off the use of bright colours in the dashboard and map.\newline
	{\bf Rationale:} Highly saturated colours are hard on the eyes and can cause visual fatigue when viewing for extended periods of time [8].
\end{enumerate}
% End SubSubSection

% End SubSection

\subsection{Performance Requirements}
\label{sub:performance_requirements}
% Begin SubSection

\subsubsection{Speed and Latency Requirements}
\label{ssub:speed_and_latency_requirements}
% Begin SubSubSection
\begin{enumerate}[{PR-SL}1. ]
	\item The system shall ensure the latency of the data appearing on the City Operator dashboard does not exceed 1.5 seconds.\newline
	{\bf Rationale:} Any delay in getting information to a city official directly increases the risk to public safety [4].
	\item The system shall authenticate a user and load their specific role-based dashboard in less than 3 seconds after the user clicks login.\newline
	{\bf Rationale:} For City Operators, they would need to respond to events in a fast manner, as a delay in logging in during a crisis could delay critical decision-making [4].
	\item The Public API shall return the requested data within 1 second.\newline
	{\bf Rationale:} Fast access to public data is essential for transparency and third-party developers. If the public-facing tools are slow, users are less likely to use the system for daily planning [9].
\end{enumerate}
% End SubSubSection

\subsubsection{Safety-Critical Requirements}
\label{ssub:safety_critical_requirements}
% Begin SubSubSection
\begin{enumerate}[{PR-SC}1. ]
	\item N/A
\end{enumerate}
% End SubSubSection

\subsubsection{Precision or Accuracy Requirements}
\label{ssub:precision_or_accuracy_requirements}
% Begin SubSubSection
\begin{enumerate}[{PR-PA}1. ]
	\item The system shall store and display all numerical data (e.g., temperature, humidity, and particulate matter) to at least two decimal places.\newline
	{\bf Rationale:} High-resolution data is needed to identify subtle environmental trends. Rounding to the nearest whole number can hide small changes in events, which could prevent the alert of an event. This level of detail is necessary to detect the beginning trends before they cross the thresholds [4].
	\item The system shall represent sensor locations and alert markers on the geographical map with coordinate precision of at least 1 meter.\newline
	{\bf Rationale:} As stated in Section 2.1, the system must allow officials to pinpoint exact locations of hazards. Low-precision coordinates could place an alert on the wrong area, leading to responders to the incorrect location [10].
\end{enumerate}
% End SubSubSection

\subsubsection{Reliability and Availability Requirements}
\label{ssub:reliability_and_availability_requirements}
% Begin SubSubSection
\begin{enumerate}[{PR-RA}1. ]
	\item The system shall maintain an availability of 99.99\%, limiting the system updates and maintenance to no more than an hour a year.\newline
	{\bf Rationale:} As the system is responsible for detecting hazards, it must be readily available. Despite the industry standard being an uptime of 99.99\% [11], it is not attainable even for most services. Most cloud service providers' setups could achieve 99.99\% availability [11].
	\item If the system or database experiences an unplanned failure, it shall automatically recover and restore core services within 120 seconds, without manual intervention.\newline
	{\bf Rationale:} Disaster recovery aims to minimize business disruption by restoring service quickly after failures. This requirement sets an explicit RTO to reduce the operational "blind spot" during time-sensitive environmental incidents [12].
\end{enumerate}
% End SubSubSection

\subsubsection{Robustness or Fault-Tolerance Requirements}
\label{ssub:robustness_or_fault_tolerance_requirements}
% Begin SubSubSection
\begin{enumerate}[{PR-RFT}1. ]
	\item The system shall store all incoming sensor data in a local cache for up to 48 hours during a network or internet failure. Upon reconnection, the system shall automatically upload this data to the main database in chronological order.\newline
	{\bf Rationale:} Smart-city sensor networks must tolerate intermittent connectivity, including extended outages during major emergencies. IoT architectures commonly support offline operation by using local caching so telemetry is not lost and can be reliably forwarded when connectivity returns [13].
	\item The system shall continue to provide core functions, even if non-critical subsystems become unavailable.\newline
	{\bf Rationale:} If one part of the app breaks, the whole system should not crash. Using "graceful degradation" ensures that essential tools remain available [14].
\end{enumerate}
% End SubSubSection

\subsubsection{Capacity Requirements}
\label{ssub:capacity_requirements}
% Begin SubSubSection
\begin{enumerate}[{PR-C}1. ]
	\item The system shall support at least 50 concurrent City Operators performing dashboard actions simultaneously without data collision, inconsistent state, or system failure.\newline
	{\bf Rationale:} Multiple city operators may operate the dashboard simultaneously. The system must handle concurrent access safely to prevent conflicting updates (such as simultaneous alert edits) and ensure operational continuity under peak load [15].
\end{enumerate}
% End SubSubSection

\subsubsection{Scalability or Extensibility Requirements}
\label{ssub:scalability_or_extensibility_requirements}
% Begin SubSubSection
\begin{enumerate}[{PR-SE}1. ]
	\item The system shall be able to scale to support deployment in multiple cities of varying population sizes, with increases in the number of sensors, monitored zones, and concurrent users, without requiring an architectural redesign.\newline
	{\bf Rationale:} Designing for scalability allows the system to grow in users, sensors, and coverage while maintaining performance and avoiding major redesign [4].
\end{enumerate}
% End SubSubSection

\subsubsection{Longevity Requirements}
\label{ssub:longevity_requirements}
% Begin SubSubSection
\begin{enumerate}[{PR-L}1. ]
	\item The system shall be designed with a minimum operational lifespan of at least 10 years.\newline
	{\bf Rationale:} This system is a large investment for a city, so to provide a high return on investment and ensure safety, the software must remain viable and maintainable without requiring a complete replacement [4].
\end{enumerate}
% End SubSubSection

% End SubSection

\subsection{Operational and Environmental Requirements}
\label{sub:operational_and_environmental_requirements}
% Begin SubSection

\subsubsection{Expected Physical Environment}
\label{ssub:expected_physical_environment}
% Begin SubSubSection
\begin{enumerate}[{OE-EPE}1. ]
	\item N/A
\end{enumerate}
% End SubSubSection

\subsubsection{Requirements for Interfacing with Adjacent Systems}
\label{ssub:requirements_for_interfacing_with_adjacent_systems}
% Begin SubSubSection
\begin{enumerate}[{OE-IA}1. ]
	\item The system must implement a read-only REST API endpoint allowing third-party and public access to non-sensitive aggregated environmental data.\newline
	{\bf Rationale:} It is important for the system to be able to get important safety information out to the general public, and providing this API to other services will facilitate this communication [4].
	\item The system should be able to reliably receive and process data from designated sensors.\newline
	{\bf Rationale:} The application must be able to receive sensor data in order to properly activate alerts and notify third party systems [4].
\end{enumerate}
% End SubSubSection

\subsubsection{Productization Requirements}
\label{ssub:productization_requirements}
% Begin SubSubSection
\begin{enumerate}[{OE-P}1. ]
	\item The application should be packaged in a zip file containing a README.\newline
	{\bf Rationale:} The application should be quick to download and be easy to install and set up [4].
\end{enumerate}
% End SubSubSection

\subsubsection{Release Requirements}
\label{ssub:release_requirements}
% Begin SubSubSection
\begin{enumerate}[{OE-R}1. ]
	\item The application should be compatible with the last 2 major versions of the Windows operating system.\newline
	{\bf Rationale:} The system should be compatible with the most common desktop operating system worldwide [16].
\end{enumerate}
% End SubSubSection

% End SubSection

\subsection{Maintainability and Support Requirements}
\label{sub:maintainability_and_support_requirements}
% Begin SubSection

\subsubsection{Maintenance Requirements}
\label{ssub:maintenance_requirements}
% Begin SubSubSection
\begin{enumerate}[{MS-M}1. ]
	\item The system must be designed for modular component (microservices) upgrades independent of the entire system.\newline
	{\bf Rationale:} A long maintenance window can result in missed alerts during critical environmental events. This will reduce risk and overall system downtime by completing individual software patches [17].
\end{enumerate}
% End SubSubSection

\subsubsection{Supportability Requirements}
\label{ssub:supportability_requirements}
% Begin SubSubSection
\begin{enumerate}[{MS-S}1. ]
	\item The system must show application health metrics such as API latency, error rates, traffic and host saturation for all administrators.\newline
	{\bf Rationale:} Administrators must be able to resolve data ingestion failures and offline issues immediately [18].
\end{enumerate}
% End SubSubSection

\subsubsection{Adaptability Requirements}
\label{ssub:adaptability_requirements}
% Begin SubSubSection
\begin{enumerate}[{MS-A}1. ]
	\item The system architecture must support the addition of new data ingestion for different environmental metric types.\newline
	{\bf Rationale:} Environments are constantly changing and there will be new climate trends to be monitored. The system must be sustainable long-term and support easy additions without changing core logic [19].
\end{enumerate}
% End SubSubSection

% End SubSection

\subsection{Security Requirements}
\label{sub:security_requirements}
% Begin SubSection

\subsubsection{Access Requirements}
\label{ssub:access_requirements}
% Begin SubSubSection
\begin{enumerate}[{SR-AC}1. ]
	\item The system should only allow administrators to modify authenticated users and alert rules.\newline
	{\bf Rationale:} There should be a way to add, manage, and delete users so that city operators can be successfully authenticated without public users having the same abilities. A Role-Based Access Control model [20] fits this need by defining a hierarchy of authentication levels through various roles.
	\item The system should only allow city operators and system administrators to view the dashboard.\newline
	{\bf Rationale:} Public users should not be permitted to access all data through the system dashboard since this could reveal sensitive data to the public. A Role-Based Access Control model [20] fits this need by defining a hierarchy of authentication levels through various roles.
\end{enumerate}
% End SubSubSection

\subsubsection{Integrity Requirements}
\label{ssub:integrity_requirements}
% Begin SubSubSection
\begin{enumerate}[{SR-INT}1. ]
	\item Data should be encrypted while in transit.\newline
	{\bf Rationale:} Sensitive sensor data could be read if not encrypted leading to potential security risks and data leaks [21]. By keeping data encrypted, users and clients can trust the product to keep their data safe which will improve trust.
\end{enumerate}
% End SubSubSection

\subsubsection{Privacy Requirements}
\label{ssub:privacy_requirements}
% Begin SubSubSection
\begin{enumerate}[{SR-P}1. ]
	\item The system shall not collect, process, or store any personally identifiable information (PII).\newline
	{\bf Rationale:} This will ensure compliance with privacy regulations while reducing legal, ethical, and security risks [21].
	\item The public API shall only expose non-sensitive environmental data aggregated by city zone.\newline
	{\bf Rationale:} Revealing precise sensor data to the public could reveal the location of sensors used in the city [21]. Giving aggregated data balances transparency with the security of the system.
\end{enumerate}
% End SubSubSection

\subsubsection{Audit Requirements}
\label{ssub:audit_requirements}
% Begin SubSubSection
\begin{enumerate}[{SR-AU}1. ]
	\item The system shall maintain an immutable log of all significant events, including alert triggers, alert acknowledgement and alert rule modifications.\newline
	{\bf Rationale:} Maintaining an unmodifiable log of all events can be used to audit the system and to track administrative changes. This follows the guidelines for cybersecurity set out by the Government of Canada [21].
\end{enumerate}
% End SubSubSection

\subsubsection{Immunity Requirements}
\label{ssub:immunity_requirements}
% Begin SubSubSection
\begin{enumerate}[{SR-IM}1. ]
	\item Public-facing interfaces shall enforce rate limits to protect against denial-of-service attacks and excessive data scraping.\newline
	{\bf Rationale:} Limiting the rate public users can access the API should reduce the strain on servers and limit the risk of a successful denial-of-service attack. This method of resisting denial-of-service attacks is outlined in ITSG-33 (Security Control SC-5) [21].
\end{enumerate}
% End SubSubSection

% End SubSection

\subsection{Cultural and Political Requirements}
\label{sub:cultural_and_political_requirements}
% Begin SubSection

\subsubsection{Cultural Requirements}
\label{ssub:cultural_requirements}
% Begin SubSubSection
\begin{enumerate}[{CP-C}1. ]
	\item The system shall support all official languages of the country it is operating in.\newline
	{\bf Rationale:} National language laws state that for government systems all official languages must be available [22].
	\item The app and system will not use any offensive or hurtful language towards any individual or people group based on their; citizenship, gender, sexual orientation, religion, political views, disability status, or any other common identifier.\newline
	{\bf Rationale:} Hate speech laws must be followed and all users should feel safe when using the app [23].
\end{enumerate}
% End SubSubSection

\subsubsection{Political Requirements}
\label{ssub:political_requirements}
% Begin SubSubSection
\begin{enumerate}[{CP-P}1. ]
	\item The public facing app should not use fear inducing terms when describing alerts. Only scientifically backed terms will be used.\newline
	{\bf Rationale:} To not create panic or fear when a fire or pollution alert is out [24].
\end{enumerate}
% End SubSubSection

% End SubSection

\subsection{Legal Requirements}
\label{sub:legal_requirements}
% Begin SubSection

\subsubsection{Compliance Requirements}
\label{ssub:compliance_requirements}
% Begin SubSubSection
\begin{enumerate}[{LR-COMP}1. ]
	\item The app will not process any raw audio data. IoT devices will process audio and never transmit raw audio streams only sound levels.\newline
	{\bf Rationale:} Live audio streaming without consent is wiretapping under the Personal Information Protection and Electronic Documents Act and the Criminal Code of Canada Section 184 [25] [26].
	\item The app's public facing app will meet the AODA accessibility requirements.\newline
	{\bf Rationale:} The app must meet the accessibility requirements by law [27].
	\item All app data will be stored on servers inside the country of origin.\newline
	{\bf Rationale:} This is the law on how municipalities have to collect and store data [27].
	\item The app will contain immutable audit logs and have frequent backups all while being secure for a minimum of 7 years.\newline
	{\bf Rationale:} The law states municipalities have to retain records in a secure manner to fulfill freedom of information requests and audits [28].
	\item Environmental action taken by a user in the dashboard when responding to an event has to be legal.\newline
	{\bf Rationale:} All environmental actions regarding pollution, toxicity, and waste, need to follow the Canadian Environmental Protection Act [29].
\end{enumerate}
% End SubSubSection

\subsubsection{Standards Requirements}
\label{ssub:standards_requirements}
% Begin SubSubSection
\begin{enumerate}[{LR-STD}1. ]
	\item The app will follow the W3C WCAG 2.1 Accessibility Guidelines.\newline
	{\bf Rationale:} This is to meet AODA web standards [30].
\end{enumerate}
% End SubSubSection

% End SubSection

% End Section

\section{Innovative Feature}
\label{sec:innovative_feature}
% Begin Section
The chosen innovative feature we will be implementing is a public interface for our SCEMAS application that will allow any public user to view dashboards which will visualize all of the publicly available environmental metrics available from the REST API. This feature will benefit the product by increasing the number of users for our application. Since the REST API will be mainly used by third-party developers, we want to allow regular public users to also view and be informed of environmental issues. We have also come up with other innovative features, as listed below:

\begin{itemize}
	\item Provide access to external resources that guide users on how to navigate specific environmental concerns whenever a warning is issued.
	\item The dashboard map uses highlighted areas to emphasize zones that require extra precaution, such as school zones and residential neighbourhoods.
	\item An integrated RAG (Retrieval-Augmented Generation) model that leverages city documentation, such as covering water treatment, waste management, snow removal, and emergency services, to provide intelligent suggestions. Users can review these insights before choosing to take action via dedicated buttons, such as calling 911 or routing the issue to the appropriate department.
	\item The system generates insights based on real-time weather data. For example, it can identify heavy snow accumulation and automatically recalculate and transmit optimized routes to snow plow operators.
	\item Administrators can utilize biometric authentication, including fingerprint and retinal scans.
	\item Rather than simply displaying current sensor data, the platform predicts potential risks by analyzing trends. If particulate matter concentration begins to rise rapidly toward the 100 $\mu$g/m$^3$ threshold, the system will preemptively alert operators before the limit is officially breached.
	\item Notification settings allow organizations to assign specific employees to monitor particular alert types, such as air quality, or to oversee designated geographic regions.
	\item The interface fully supports both light and dark modes to accommodate different lighting environments and user preferences.
\end{itemize}

% End Section

\appendix
\section{Division of Labour}
\label{sec:division_of_labour}
% Begin Section
This sheet indicates the contributions of each team member and is signed by all team members. \newline

\noindent
Khan, Saqib
\begin{itemize}
	\item Introduction 
	\item System Diagram 
	\item Use-case Diagram as a group
	\item 2.4 Constraints
	\item Brainstorming requirements, business events
	\item BE2. Modify Existing Alert Rule
	\item 5 Non-Functional requirements discussion
	\item 5.1 Look and Feel Requirements contribution 
	\item 5.3 Performance Requirements
	\item Discussed Innovative feature 
\end{itemize}

\includegraphics[width=0.3\textwidth]{images/Saqib}

\noindent
Abdullah, Suzanne
\begin{itemize}
	\item 2.3 User Characteristics
	\item System Diagram contribution
	\item Use-case Diagram with group
	\item BE6. Checking Audit Log
	\item 5.1 Look and Feel Requirements
	\item 5.4 Operational and Environmental Requirements
	\item Innovative feature main idea
	\item Brainstorming requirements, business events
\end{itemize}

\includegraphics[width=0.3\textwidth]{images/Suzanne}

\noindent
Olejniczak, David
\begin{itemize}
	\item 1.4 References + formatting references in Non-Functional Requirements
	\item 2.1 Product Perspective Paragraph
	\item Use-case Diagram as a group
	\item 2.4 Assumptions
	\item BE1. Creating an Alert Rule
	\item BE5. Public API Request to Access Data
	\item BE7. User authentication
	\item 5.7 Cultural and Political Requirements
	\item 5.8 Legal Requirements
	\item Brainstormed Innovative feature with group
\end{itemize}

\includegraphics[width=0.3\textwidth]{images/David}

\noindent
Lai, Vanessa
\begin{itemize}
	\item 1.2 Scope
	\item BE4. View Dashboard
	\item 5.2 Usability and Humanity Requirements
	\item 5.5 Maintainability and Support Requirements
	\item State Diagram
	\item Use-case Diagram as a group
	\item Brainstormed Innovative feature with group
	\item 6. Innovative Feature
\end{itemize}

\includegraphics[width=0.3\textwidth]{images/Vanessa}

\noindent
Buehlmann, Lukas
\begin{itemize}
	\item 1.5 Overview
	\item 2.2 Product Functions
	\item Use-case Diagram as a group
	\item BE3. Sunlight is very high on the UV index
	\item 5.6 Security Requirements
	\item Brainstormed Innovative Ideas with group
	\item Reviewed and edited final document
\end{itemize}

\includegraphics[width=0.3\textwidth]{images/Lukas}

% End Section

%\newpage
%\section*{IMPORTANT NOTES}
%\begin{itemize}
%	\item Be sure to include all sections of the template in your document regardless whether you have something to write for each or not
%	\begin{itemize}
%		\item If you do not have anything to write in a section, indicate this by the \emph{N/A}, \emph{void}, \emph{none}, etc.
%	\end{itemize}
%	\item Uniquely number each of your requirements for easy identification and cross-referencing
%	\item Highlight terms that are defined in Section~1.3 (\textbf{Definitions, Acronyms, and Abbreviations}) with \textbf{bold}, \emph{italic} or \underline{underline}
%	\item For Deliverable 1, please highlight, in some fashion, all (you may have more than one) creative and innovative features. Your creative and innovative features will generally be described in Section~2.2 (\textbf{Product Functions}), but it will depend on the type of creative or innovative features you are including.
%\end{itemize}


\end{document}
%------------------------------------------------------------------------------